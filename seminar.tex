\documentclass{beamer}
\usepackage{babel}
\usepackage[utf8]{inputenc}
\usepackage[T1]{fontenc}
\usepackage[sfdefault]{FiraSans}
\usepackage{graphicx}
\usepackage{stmaryrd}
\usepackage{xcolor}
\usepackage{tikz}
\usepackage{proof}
\usepackage{listings}
\usepackage{amsmath}
\usetheme{Copenhagen}
\setbeamertemplate{navigation symbols}{}
\setbeamertemplate{headline}{}
\AtBeginSection[ ]
{
\begin{frame}{Outline}
    \tableofcontents[currentsection, hideallsubsections]
\end{frame}
}
\title{Rewriting Logic \& Maude}
\author{Niccolò Piazzesi}
\institute{
    Università degli studi di Pisa
}
\titlegraphic{\includegraphics[width=2cm]{img/logo.png}}
\begin{document}

\frame{\titlepage}

\begin{frame}{Outline}
    \tableofcontents[hideallsubsections]
\end{frame}
\section{Introduction and main concepts}
\begin{frame}
    \frametitle{What is a concurrent system?}
    Modelling concurrent system is one of the most studied problems in Computer Science

    \bigskip
    \pause
    Many proposed answers:\begin{itemize}
        \item Petri Nets 
        \item CCS
        \item CSP 
        \item Actors
        \item $\cdots$
    \end{itemize}
\end{frame}
\begin{frame}
    \frametitle{The need for unification}
    \pause
        \begin{block}{External fragmentation}
            Hard to relate different approaches, each with their own concepts, models and issues
        \end{block}
    \pause 
    \begin{block}{Internal fragmentation}
        Sometimes, fragmentation appears also within a specific approach (e.g. how can we unify operational and denotational semantics of CCS?)
    \end{block}
    \pause
    \begin{block}{Concurrency in other areas}
        A related problem is the integration of concurrency with other paradigms ( OO, Functional, ...)
        without using complex \emph{ad hoc} solutions.
    \end{block}
\end{frame}

\begin{frame}
    
    Rewriting logic aim to resolve these issues with a two sides approach
    \pause
    \begin{block}{Computational side}
        Computationally, rewriting logic is a \emph{semantic framework} where many different 
        models and languages can be represented and executed as \textbf{rewrite theories}. 
    \end{block}
    \pause
    \begin{block}{Logical side}
        Rewriting logic  is also a \emph{logical framework}, a model within which 
        many other logics and deduction procedures  can be modeled and reasoned about.        
    \end{block}
\end{frame}
\section{Foundation}
\section{Semantic model}
\section{Maude}
\section{Applications}
\end{document}